% Options for packages loaded elsewhere
% Options for packages loaded elsewhere
\PassOptionsToPackage{unicode}{hyperref}
\PassOptionsToPackage{hyphens}{url}
\PassOptionsToPackage{dvipsnames,svgnames,x11names}{xcolor}
%
\documentclass[
  brazilian,
  letterpaper,
  DIV=11,
  numbers=noendperiod,
  oneside]{scrartcl}
\usepackage{xcolor}
\usepackage[top=25mm,bottom=25mm,left=25mm,right=25mm]{geometry}
\usepackage{amsmath,amssymb}
\setcounter{secnumdepth}{5}
\usepackage{iftex}
\ifPDFTeX
  \usepackage[T1]{fontenc}
  \usepackage[utf8]{inputenc}
  \usepackage{textcomp} % provide euro and other symbols
\else % if luatex or xetex
  \usepackage{unicode-math} % this also loads fontspec
  \defaultfontfeatures{Scale=MatchLowercase}
  \defaultfontfeatures[\rmfamily]{Ligatures=TeX,Scale=1}
\fi
\usepackage{lmodern}
\ifPDFTeX\else
  % xetex/luatex font selection
\fi
% Use upquote if available, for straight quotes in verbatim environments
\IfFileExists{upquote.sty}{\usepackage{upquote}}{}
\IfFileExists{microtype.sty}{% use microtype if available
  \usepackage[]{microtype}
  \UseMicrotypeSet[protrusion]{basicmath} % disable protrusion for tt fonts
}{}
\makeatletter
\@ifundefined{KOMAClassName}{% if non-KOMA class
  \IfFileExists{parskip.sty}{%
    \usepackage{parskip}
  }{% else
    \setlength{\parindent}{0pt}
    \setlength{\parskip}{6pt plus 2pt minus 1pt}}
}{% if KOMA class
  \KOMAoptions{parskip=half}}
\makeatother
% Make \paragraph and \subparagraph free-standing
\makeatletter
\ifx\paragraph\undefined\else
  \let\oldparagraph\paragraph
  \renewcommand{\paragraph}{
    \@ifstar
      \xxxParagraphStar
      \xxxParagraphNoStar
  }
  \newcommand{\xxxParagraphStar}[1]{\oldparagraph*{#1}\mbox{}}
  \newcommand{\xxxParagraphNoStar}[1]{\oldparagraph{#1}\mbox{}}
\fi
\ifx\subparagraph\undefined\else
  \let\oldsubparagraph\subparagraph
  \renewcommand{\subparagraph}{
    \@ifstar
      \xxxSubParagraphStar
      \xxxSubParagraphNoStar
  }
  \newcommand{\xxxSubParagraphStar}[1]{\oldsubparagraph*{#1}\mbox{}}
  \newcommand{\xxxSubParagraphNoStar}[1]{\oldsubparagraph{#1}\mbox{}}
\fi
\makeatother


\usepackage{longtable,booktabs,array}
\usepackage{calc} % for calculating minipage widths
% Correct order of tables after \paragraph or \subparagraph
\usepackage{etoolbox}
\makeatletter
\patchcmd\longtable{\par}{\if@noskipsec\mbox{}\fi\par}{}{}
\makeatother
% Allow footnotes in longtable head/foot
\IfFileExists{footnotehyper.sty}{\usepackage{footnotehyper}}{\usepackage{footnote}}
\makesavenoteenv{longtable}
\usepackage{graphicx}
\makeatletter
\newsavebox\pandoc@box
\newcommand*\pandocbounded[1]{% scales image to fit in text height/width
  \sbox\pandoc@box{#1}%
  \Gscale@div\@tempa{\textheight}{\dimexpr\ht\pandoc@box+\dp\pandoc@box\relax}%
  \Gscale@div\@tempb{\linewidth}{\wd\pandoc@box}%
  \ifdim\@tempb\p@<\@tempa\p@\let\@tempa\@tempb\fi% select the smaller of both
  \ifdim\@tempa\p@<\p@\scalebox{\@tempa}{\usebox\pandoc@box}%
  \else\usebox{\pandoc@box}%
  \fi%
}
% Set default figure placement to htbp
\def\fps@figure{htbp}
\makeatother



\ifLuaTeX
\usepackage[bidi=basic]{babel}
\else
\usepackage[bidi=default]{babel}
\fi
% get rid of language-specific shorthands (see #6817):
\let\LanguageShortHands\languageshorthands
\def\languageshorthands#1{}


\setlength{\emergencystretch}{3em} % prevent overfull lines

\providecommand{\tightlist}{%
  \setlength{\itemsep}{0pt}\setlength{\parskip}{0pt}}



 


\usepackage{booktabs}
\usepackage{longtable}
\usepackage{array}
\usepackage{multirow}
\usepackage{wrapfig}
\usepackage{float}
\usepackage{colortbl}
\usepackage{pdflscape}
\usepackage{tabu}
\usepackage{threeparttable}
\usepackage{threeparttablex}
\usepackage[normalem]{ulem}
\usepackage{makecell}
\usepackage{xcolor}
\usepackage{caption}
\usepackage{anyfontsize}
\KOMAoption{captions}{tableheading}
\makeatletter
\@ifpackageloaded{caption}{}{\usepackage{caption}}
\AtBeginDocument{%
\ifdefined\contentsname
  \renewcommand*\contentsname{Índice}
\else
  \newcommand\contentsname{Índice}
\fi
\ifdefined\listfigurename
  \renewcommand*\listfigurename{Lista de Figuras}
\else
  \newcommand\listfigurename{Lista de Figuras}
\fi
\ifdefined\listtablename
  \renewcommand*\listtablename{Lista de Tabelas}
\else
  \newcommand\listtablename{Lista de Tabelas}
\fi
\ifdefined\figurename
  \renewcommand*\figurename{Figura}
\else
  \newcommand\figurename{Figura}
\fi
\ifdefined\tablename
  \renewcommand*\tablename{Tabela}
\else
  \newcommand\tablename{Tabela}
\fi
}
\@ifpackageloaded{float}{}{\usepackage{float}}
\floatstyle{ruled}
\@ifundefined{c@chapter}{\newfloat{codelisting}{h}{lop}}{\newfloat{codelisting}{h}{lop}[chapter]}
\floatname{codelisting}{Listagem}
\newcommand*\listoflistings{\listof{codelisting}{Lista de Listagens}}
\makeatother
\makeatletter
\makeatother
\makeatletter
\@ifpackageloaded{caption}{}{\usepackage{caption}}
\@ifpackageloaded{subcaption}{}{\usepackage{subcaption}}
\makeatother
\makeatletter
\@ifpackageloaded{sidenotes}{}{\usepackage{sidenotes}}
\@ifpackageloaded{marginnote}{}{\usepackage{marginnote}}
\makeatother
\usepackage{bookmark}
\IfFileExists{xurl.sty}{\usepackage{xurl}}{} % add URL line breaks if available
\urlstyle{same}
\hypersetup{
  pdftitle={Análise de séries temporais},
  pdfauthor={Rafael Sebastião Arocho e Márcio Antonio Vieira},
  pdflang={pt-br},
  colorlinks=true,
  linkcolor={blue},
  filecolor={Maroon},
  citecolor={Blue},
  urlcolor={Blue},
  pdfcreator={LaTeX via pandoc}}


\title{Análise de séries temporais}
\usepackage{etoolbox}
\makeatletter
\providecommand{\subtitle}[1]{% add subtitle to \maketitle
  \apptocmd{\@title}{\par {\large #1 \par}}{}{}
}
\makeatother
\subtitle{Disciplina: Métodos Estatísticos de Previsão}
\author{Rafael Sebastião Arocho e Márcio Antonio Vieira}
\date{19/11/2025}
\begin{document}
\maketitle


\newpage{}

\section{Visão Geral dos Dados}\label{visuxe3o-geral-dos-dados}

Para esta análise, utilizamos o dataset \textbf{\emph{House Property
Sales Time Series}}.

\begin{itemize}
\item
  \textbf{Fonte dos dados:}
  \href{https://www.kaggle.com/datasets/htagholdings/property-sales?select=ma_lga_12345.csv}{Kaggle.}
\item
  \textbf{Período:} 2007 a 2019
\item
  \textbf{Frequência:} Trimestral (\(n = 347\) observações)
\item
  \textbf{Variável Resposta (}\(Y_t\)):

  \begin{itemize}
  \tightlist
  \item
    \emph{MA}: Preço (\$) mediano de casas e unidades habitacionais
  \end{itemize}
\end{itemize}

\begin{table}

\caption{\label{tbl-resumo-serie}resumo dos dados}

\centering{

\centering
\resizebox{\ifdim\width>\linewidth\linewidth\else\width\fi}{!}{
\begin{tabular}{lcccc}
\toprule
\textbf{Characteristic} & \textbf{N} & \makecell[c]{\textbf{Overall}\ \ \\N = 347} & \makecell[c]{\textbf{house}\ \ \\N = 200} & \makecell[c]{\textbf{unit}\ \ \\N = 147}\\
\midrule
MA & 347 & 507,744 (427,623, 628,423) & 584,932 (484,806, 771,248) & 425,922 (339,125, 535,063)\\
bedrooms & 347 &  &  & \\
\hspace{1em}1 &  & 48 (14\%) & 0 (0\%) & 48 (33\%)\\
\hspace{1em}2 &  & 99 (29\%) & 49 (25\%) & 50 (34\%)\\
\hspace{1em}3 &  & 100 (29\%) & 51 (26\%) & 49 (33\%)\\
\addlinespace
\hspace{1em}4 &  & 51 (15\%) & 51 (26\%) & 0 (0\%)\\
\hspace{1em}5 &  & 49 (14\%) & 49 (25\%) & 0 (0\%)\\
\bottomrule
\multicolumn{5}{l}{\rule{0pt}{1em}\textsuperscript{1} Median (Q1, Q3); n (\%)}\\
\end{tabular}}

}

\end{table}%

\newpage{}

\section{Visualização da Série}\label{visualizauxe7uxe3o-da-suxe9rie}

\begin{figure}

\begin{minipage}{0.50\linewidth}

\begin{figure}[H]

{\centering \pandocbounded{\includegraphics[keepaspectratio]{trabalho_mep_files/figure-pdf/series-chart-1.pdf}}

}

\subcaption{serie completa}

\end{figure}%

\end{minipage}%
%
\begin{minipage}{0.50\linewidth}

\begin{figure}[H]

{\centering \pandocbounded{\includegraphics[keepaspectratio]{trabalho_mep_files/figure-pdf/series-chart-2.pdf}}

}

\subcaption{serie decomposta}

\end{figure}%

\end{minipage}%
\newline
\begin{minipage}{0.50\linewidth}
Distribuições dos dados\end{minipage}%

\end{figure}%

Observamos que será necessário separar as series em tipo de moradia e
número de quartos para simplificar o ajuste.\\
Parece haver tendência em praticamente todas as séries decompostas.
Portanto, vamos utilizar diferenciação de nível um e dois para ter outro
olhar sobre as séries.

\begin{figure*}[H]

\begin{figure}[H]

{\centering \pandocbounded{\includegraphics[keepaspectratio]{trabalho_mep_files/figure-pdf/unnamed-chunk-1-1.pdf}}

}

\caption{diferenciação de primeiro nível}

\end{figure}%

\end{figure*}

\begin{figure*}[H]

\begin{figure}[H]

{\centering \pandocbounded{\includegraphics[keepaspectratio]{trabalho_mep_files/figure-pdf/unnamed-chunk-1-2.pdf}}

}

\caption{diferenciação de segundo nível}

\end{figure}%

\end{figure*}

comportamentos diferenciados

Observando as séries com as diferenciações aplicadas vamos seguir a
análise com os preços das casas de 5 quartos.

\newpage{}

\section{Ajuste de modelos}\label{ajuste-de-modelos}

Vamos observar o ACF e o PACF para os dados filtrados com o critério
anterior (casas de 5 quartos).\\
Verificaremos tanto para a série não diferenciada quanto diferenciada
para um e dois retardos.

\begin{figure}[H]

{\centering \pandocbounded{\includegraphics[keepaspectratio]{trabalho_mep_files/figure-pdf/acf_pacf primeiro modelo-1.pdf}}

}

\caption{serie original}

\end{figure}%

\begin{figure}[H]

{\centering \pandocbounded{\includegraphics[keepaspectratio]{trabalho_mep_files/figure-pdf/acf_pacf primeiro modelo-2.pdf}}

}

\caption{acf primeira diferenciação}

\end{figure}%

\begin{figure}[H]

{\centering \pandocbounded{\includegraphics[keepaspectratio]{trabalho_mep_files/figure-pdf/acf_pacf primeiro modelo-3.pdf}}

}

\caption{pacf primeira diferenciação}

\end{figure}%

\begin{figure}[H]

{\centering \pandocbounded{\includegraphics[keepaspectratio]{trabalho_mep_files/figure-pdf/acf_pacf primeiro modelo-4.pdf}}

}

\caption{acf segunda diferenciação}

\end{figure}%

\begin{figure}[H]

{\centering \pandocbounded{\includegraphics[keepaspectratio]{trabalho_mep_files/figure-pdf/acf_pacf primeiro modelo-5.pdf}}

}

\caption{pacf segunda diferenciação}

\end{figure}%

Parecemos ter um comportamento de ondas senóides no gráfico de ACF e um
comportamento irregular no PACF que pode indicar sazonalidade. Vamos
ajustar de ínicio um modelo ARIMA(2, 1, 0). Após o ajuste inicial
faremos a sobrefixação dos modelos para identificar o melhor modelo
através da avaliação das métricas de AIC

\begin{table}
\caption*{
{\fontsize{20}{25}\selectfont  ARIMA(MA \textasciitilde{} pdq(2, 1, 0) + PDQ(0, 0, 0))\fontsize{12}{15}\selectfont }
} 
\fontsize{12.0pt}{14.0pt}\selectfont
\begin{tabular*}{\linewidth}{@{\extracolsep{\fill}}lrlrrrr}
\toprule
type & bedrooms & term & estimate & std.error & statistic & p.value \\ 
\midrule\addlinespace[2.5pt]
house & 5 & ar1 & 0.3561 & 0.1481837 & 2.4031 & 0.0208 \\ 
house & 5 & ar2 & 0.2164 & 0.1489611 & 1.4527 & 0.1537 \\ 
house & 5 & constant & 2,706.8068 & 907.0361221 & 2.9842 & 0.0047 \\ 
\bottomrule
\end{tabular*}
\end{table}

\begin{table}
\caption*{
{\fontsize{20}{25}\selectfont  ARIMA(MA \textasciitilde{} pdq(1, 1, 0) + PDQ(0, 0, 0))\fontsize{12}{15}\selectfont }
} 
\fontsize{12.0pt}{14.0pt}\selectfont
\begin{tabular*}{\linewidth}{@{\extracolsep{\fill}}lrlrrrr}
\toprule
type & bedrooms & term & estimate & std.error & statistic & p.value \\ 
\midrule\addlinespace[2.5pt]
house & 5 & ar1 & 0.4542 & 0.1361811 & 3.3350 & 0.0018 \\ 
house & 5 & constant & 3,544.1817 & 948.8267730 & 3.7353 & 0.0006 \\ 
\bottomrule
\end{tabular*}
\end{table}

\begin{table}
\caption*{
{\fontsize{20}{25}\selectfont  ARIMA(MA \textasciitilde{} pdq(1, 1, 1) + PDQ(0, 0, 0))\fontsize{12}{15}\selectfont }
} 
\fontsize{12.0pt}{14.0pt}\selectfont
\begin{tabular*}{\linewidth}{@{\extracolsep{\fill}}lrlrrrr}
\toprule
type & bedrooms & term & estimate & std.error & statistic & p.value \\ 
\midrule\addlinespace[2.5pt]
house & 5 & ar1 & 0.8113 & 0.1887735 & 4.2976 & 0.0001 \\ 
house & 5 & ma1 & -0.4850 & 0.3108357 & -1.5604 & 0.1262 \\ 
house & 5 & constant & 1,191.6759 & 452.1114131 & 2.6358 & 0.0117 \\ 
\bottomrule
\end{tabular*}
\end{table}

\begin{table}
\caption*{
{\fontsize{20}{25}\selectfont  ARIMA(MA \textasciitilde{} pdq(1, 2, 0) + PDQ(0, 0, 0))\fontsize{12}{15}\selectfont }
} 
\fontsize{12.0pt}{14.0pt}\selectfont
\begin{tabular*}{\linewidth}{@{\extracolsep{\fill}}lrlrrrr}
\toprule
type & bedrooms & term & estimate & std.error & statistic & p.value \\ 
\midrule\addlinespace[2.5pt]
house & 5 & ar1 & -0.4247 & 0.1384875 & -3.0665 & 0.0038 \\ 
\bottomrule
\end{tabular*}
\end{table}

\begin{table}
\caption*{
{\fontsize{20}{25}\selectfont  ARIMA(MA \textasciitilde{} pdq(2, 2, 0) + PDQ(0, 0, 0))\fontsize{12}{15}\selectfont }
} 
\fontsize{12.0pt}{14.0pt}\selectfont
\begin{tabular*}{\linewidth}{@{\extracolsep{\fill}}lrlrrrr}
\toprule
type & bedrooms & term & estimate & std.error & statistic & p.value \\ 
\midrule\addlinespace[2.5pt]
house & 5 & ar1 & -0.4854 & 0.1531778 & -3.1687 & 0.0029 \\ 
house & 5 & ar2 & -0.1359 & 0.1513569 & -0.8981 & 0.3744 \\ 
\bottomrule
\end{tabular*}
\end{table}

\begin{table}
\caption*{
{\fontsize{20}{25}\selectfont  ARIMA(MA \textasciitilde{} pdq(0, 2, 1) + PDQ(0, 0, 1))\fontsize{12}{15}\selectfont }
} 
\fontsize{12.0pt}{14.0pt}\selectfont
\begin{tabular*}{\linewidth}{@{\extracolsep{\fill}}lrlrrrr}
\toprule
type & bedrooms & term & estimate & std.error & statistic & p.value \\ 
\midrule\addlinespace[2.5pt]
house & 5 & ma1 & -0.4497 & 0.1546287 & -2.9082 & 0.0058 \\ 
house & 5 & sma1 & -0.7563 & 0.2168235 & -3.4881 & 0.0012 \\ 
\bottomrule
\end{tabular*}
\end{table}

\begin{table}
\caption*{
{\fontsize{20}{25}\selectfont  ARIMA(MA \textasciitilde{} pdq(1, 2, 1) + PDQ(0, 0, 1))\fontsize{12}{15}\selectfont }
} 
\fontsize{12.0pt}{14.0pt}\selectfont
\begin{tabular*}{\linewidth}{@{\extracolsep{\fill}}lrlrrrr}
\toprule
type & bedrooms & term & estimate & std.error & statistic & p.value \\ 
\midrule\addlinespace[2.5pt]
house & 5 & ar1 & 0.0001 & 0.2951405 & 0.0005 & 0.9996 \\ 
house & 5 & ma1 & -0.4497 & 0.2777205 & -1.6194 & 0.1130 \\ 
house & 5 & sma1 & -0.7567 & 0.2219684 & -3.4088 & 0.0015 \\ 
\bottomrule
\end{tabular*}
\end{table}

Podemos, também, analisar os modelos com base no AIC:

\begin{table}

\caption{\label{tbl-comparacao-aic}}

\centering{

\caption*{
{\fontsize{20}{25}\selectfont  Ranking de Modelos por AIC\fontsize{12}{15}\selectfont }
} 
\fontsize{12.0pt}{14.0pt}\selectfont
\begin{tabular*}{\linewidth}{@{\extracolsep{\fill}}lr}
\toprule
.model & AIC \\ 
\midrule\addlinespace[2.5pt]
ARIMA(MA \textasciitilde{} pdq(0, 2, 1) + PDQ(0, 0, 1)) & 832.92 \\ 
ARIMA(MA \textasciitilde{} pdq(1, 2, 1) + PDQ(0, 0, 1)) & 834.92 \\ 
ARIMA(MA \textasciitilde{} pdq(1, 2, 0) + PDQ(0, 0, 0)) & 842.28 \\ 
ARIMA(MA \textasciitilde{} pdq(2, 2, 0) + PDQ(0, 0, 0)) & 843.48 \\ 
ARIMA(MA \textasciitilde{} pdq(2, 1, 0) + PDQ(0, 0, 0)) & 859.72 \\ 
ARIMA(MA \textasciitilde{} pdq(1, 1, 1) + PDQ(0, 0, 0)) & 859.73 \\ 
ARIMA(MA \textasciitilde{} pdq(1, 1, 0) + PDQ(0, 0, 0)) & 859.77 \\ 
\bottomrule
\end{tabular*}

}

\end{table}%

Ajustados os modelos daremos sequência na análise com o modelo
ARIMA(0,2,1)(0,0,1){[}4{]} por ter apresentado significância nos
parâmetros e o menor dos valores de AIC

\begin{verbatim}

    Shapiro-Wilk normality test

data:  residuals(modelo_arima_6)$.resid
W = 0.96361, p-value = 0.1877
\end{verbatim}

\begin{verbatim}

    Box-Pierce test

data:  residuals(modelo_arima_6)$.resid
X-squared = 9.3267, df = 12, p-value = 0.6748
\end{verbatim}

\begin{table}

\caption{\label{tbl-analise-residuos}{[}``teste de normalidade de
shapiro'',``teste de box-pierce para independencia''{]}}

\centering{

\begin{figure}
\centering
\pandocbounded{\includegraphics[keepaspectratio]{trabalho_mep_files/figure-pdf/tbl-analise-residuos-1.pdf}}
\caption{distribuição dos residuos}
\end{figure}

}

\end{table}%

Os testes nos mostram que os residuos apresentam normalidade e o teste
de box-pierce não rejeita a hipótese de independencia, o que é um
indicador que o ajuste está adequado.

\newpage{}

\subsection{Previsão}\label{previsuxe3o}

Vamos utilizar o modelo escolhido para fazer uma previsão 6 passos a
frente

\begin{figure}[H]

\centering{

\pandocbounded{\includegraphics[keepaspectratio]{trabalho_mep_files/figure-pdf/fig-previsao-1.pdf}}

}

\caption{\label{fig-previsao}previsão 6 passos a frente}

\end{figure}%

Observamos que a previsão manteve a tendência que é apresentada durante
o tempo de treinamento mas acaba errando os valores reais pois no fim da
série existe uma mudança brusca de comportamento que não foi ajustada no
modelo.

\newpage{}

\subsection{Modelo de alisamento
exponencial}\label{modelo-de-alisamento-exponencial}

Seguindo a mesma lógica dos ajustes anteriores faremos o ajuste de
modelo de alisamento exponencial. Por não termos identificado
sazonalidade utilizaremos alisamento exponencial de Holt-Winters aditivo
por haver tendência e sazonalidade constante na série.

\begin{figure}[H]

\centering{

\pandocbounded{\includegraphics[keepaspectratio]{trabalho_mep_files/figure-pdf/fig-holt_winters-1.pdf}}

}

\caption{\label{fig-holt_winters}Previsão ajuste de Holt-Winters
aditivo}

\end{figure}%




\end{document}
