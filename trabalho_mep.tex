% Options for packages loaded elsewhere
\PassOptionsToPackage{unicode}{hyperref}
\PassOptionsToPackage{hyphens}{url}
\PassOptionsToPackage{dvipsnames,svgnames,x11names}{xcolor}
%
\documentclass[
]{scratcl}

\usepackage{amsmath,amssymb}
\usepackage{iftex}
\ifPDFTeX
  \usepackage[T1]{fontenc}
  \usepackage[utf8]{inputenc}
  \usepackage{textcomp} % provide euro and other symbols
\else % if luatex or xetex
  \usepackage{unicode-math}
  \defaultfontfeatures{Scale=MatchLowercase}
  \defaultfontfeatures[\rmfamily]{Ligatures=TeX,Scale=1}
\fi
\usepackage{lmodern}
\ifPDFTeX\else  
    % xetex/luatex font selection
\fi
% Use upquote if available, for straight quotes in verbatim environments
\IfFileExists{upquote.sty}{\usepackage{upquote}}{}
\IfFileExists{microtype.sty}{% use microtype if available
  \usepackage[]{microtype}
  \UseMicrotypeSet[protrusion]{basicmath} % disable protrusion for tt fonts
}{}
\makeatletter
\@ifundefined{KOMAClassName}{% if non-KOMA class
  \IfFileExists{parskip.sty}{%
    \usepackage{parskip}
  }{% else
    \setlength{\parindent}{0pt}
    \setlength{\parskip}{6pt plus 2pt minus 1pt}}
}{% if KOMA class
  \KOMAoptions{parskip=half}}
\makeatother
\usepackage{xcolor}
\usepackage[top=25mm,bottom=25mm,left=25mm,right=25mm]{geometry}
\setlength{\emergencystretch}{3em} % prevent overfull lines
\setcounter{secnumdepth}{5}
% Make \paragraph and \subparagraph free-standing
\makeatletter
\ifx\paragraph\undefined\else
  \let\oldparagraph\paragraph
  \renewcommand{\paragraph}{
    \@ifstar
      \xxxParagraphStar
      \xxxParagraphNoStar
  }
  \newcommand{\xxxParagraphStar}[1]{\oldparagraph*{#1}\mbox{}}
  \newcommand{\xxxParagraphNoStar}[1]{\oldparagraph{#1}\mbox{}}
\fi
\ifx\subparagraph\undefined\else
  \let\oldsubparagraph\subparagraph
  \renewcommand{\subparagraph}{
    \@ifstar
      \xxxSubParagraphStar
      \xxxSubParagraphNoStar
  }
  \newcommand{\xxxSubParagraphStar}[1]{\oldsubparagraph*{#1}\mbox{}}
  \newcommand{\xxxSubParagraphNoStar}[1]{\oldsubparagraph{#1}\mbox{}}
\fi
\makeatother

\usepackage{color}
\usepackage{fancyvrb}
\newcommand{\VerbBar}{|}
\newcommand{\VERB}{\Verb[commandchars=\\\{\}]}
\DefineVerbatimEnvironment{Highlighting}{Verbatim}{commandchars=\\\{\}}
% Add ',fontsize=\small' for more characters per line
\usepackage{framed}
\definecolor{shadecolor}{RGB}{241,243,245}
\newenvironment{Shaded}{\begin{snugshade}}{\end{snugshade}}
\newcommand{\AlertTok}[1]{\textcolor[rgb]{0.68,0.00,0.00}{#1}}
\newcommand{\AnnotationTok}[1]{\textcolor[rgb]{0.37,0.37,0.37}{#1}}
\newcommand{\AttributeTok}[1]{\textcolor[rgb]{0.40,0.45,0.13}{#1}}
\newcommand{\BaseNTok}[1]{\textcolor[rgb]{0.68,0.00,0.00}{#1}}
\newcommand{\BuiltInTok}[1]{\textcolor[rgb]{0.00,0.23,0.31}{#1}}
\newcommand{\CharTok}[1]{\textcolor[rgb]{0.13,0.47,0.30}{#1}}
\newcommand{\CommentTok}[1]{\textcolor[rgb]{0.37,0.37,0.37}{#1}}
\newcommand{\CommentVarTok}[1]{\textcolor[rgb]{0.37,0.37,0.37}{\textit{#1}}}
\newcommand{\ConstantTok}[1]{\textcolor[rgb]{0.56,0.35,0.01}{#1}}
\newcommand{\ControlFlowTok}[1]{\textcolor[rgb]{0.00,0.23,0.31}{\textbf{#1}}}
\newcommand{\DataTypeTok}[1]{\textcolor[rgb]{0.68,0.00,0.00}{#1}}
\newcommand{\DecValTok}[1]{\textcolor[rgb]{0.68,0.00,0.00}{#1}}
\newcommand{\DocumentationTok}[1]{\textcolor[rgb]{0.37,0.37,0.37}{\textit{#1}}}
\newcommand{\ErrorTok}[1]{\textcolor[rgb]{0.68,0.00,0.00}{#1}}
\newcommand{\ExtensionTok}[1]{\textcolor[rgb]{0.00,0.23,0.31}{#1}}
\newcommand{\FloatTok}[1]{\textcolor[rgb]{0.68,0.00,0.00}{#1}}
\newcommand{\FunctionTok}[1]{\textcolor[rgb]{0.28,0.35,0.67}{#1}}
\newcommand{\ImportTok}[1]{\textcolor[rgb]{0.00,0.46,0.62}{#1}}
\newcommand{\InformationTok}[1]{\textcolor[rgb]{0.37,0.37,0.37}{#1}}
\newcommand{\KeywordTok}[1]{\textcolor[rgb]{0.00,0.23,0.31}{\textbf{#1}}}
\newcommand{\NormalTok}[1]{\textcolor[rgb]{0.00,0.23,0.31}{#1}}
\newcommand{\OperatorTok}[1]{\textcolor[rgb]{0.37,0.37,0.37}{#1}}
\newcommand{\OtherTok}[1]{\textcolor[rgb]{0.00,0.23,0.31}{#1}}
\newcommand{\PreprocessorTok}[1]{\textcolor[rgb]{0.68,0.00,0.00}{#1}}
\newcommand{\RegionMarkerTok}[1]{\textcolor[rgb]{0.00,0.23,0.31}{#1}}
\newcommand{\SpecialCharTok}[1]{\textcolor[rgb]{0.37,0.37,0.37}{#1}}
\newcommand{\SpecialStringTok}[1]{\textcolor[rgb]{0.13,0.47,0.30}{#1}}
\newcommand{\StringTok}[1]{\textcolor[rgb]{0.13,0.47,0.30}{#1}}
\newcommand{\VariableTok}[1]{\textcolor[rgb]{0.07,0.07,0.07}{#1}}
\newcommand{\VerbatimStringTok}[1]{\textcolor[rgb]{0.13,0.47,0.30}{#1}}
\newcommand{\WarningTok}[1]{\textcolor[rgb]{0.37,0.37,0.37}{\textit{#1}}}

\providecommand{\tightlist}{%
  \setlength{\itemsep}{0pt}\setlength{\parskip}{0pt}}\usepackage{longtable,booktabs,array}
\usepackage{calc} % for calculating minipage widths
% Correct order of tables after \paragraph or \subparagraph
\usepackage{etoolbox}
\makeatletter
\patchcmd\longtable{\par}{\if@noskipsec\mbox{}\fi\par}{}{}
\makeatother
% Allow footnotes in longtable head/foot
\IfFileExists{footnotehyper.sty}{\usepackage{footnotehyper}}{\usepackage{footnote}}
\makesavenoteenv{longtable}
\usepackage{graphicx}
\makeatletter
\newsavebox\pandoc@box
\newcommand*\pandocbounded[1]{% scales image to fit in text height/width
  \sbox\pandoc@box{#1}%
  \Gscale@div\@tempa{\textheight}{\dimexpr\ht\pandoc@box+\dp\pandoc@box\relax}%
  \Gscale@div\@tempb{\linewidth}{\wd\pandoc@box}%
  \ifdim\@tempb\p@<\@tempa\p@\let\@tempa\@tempb\fi% select the smaller of both
  \ifdim\@tempa\p@<\p@\scalebox{\@tempa}{\usebox\pandoc@box}%
  \else\usebox{\pandoc@box}%
  \fi%
}
% Set default figure placement to htbp
\def\fps@figure{htbp}
\makeatother

\usepackage{booktabs}
\usepackage{caption}
\usepackage{longtable}
\usepackage{colortbl}
\usepackage{array}
\usepackage{anyfontsize}
\usepackage{multirow}
\makeatletter
\@ifpackageloaded{caption}{}{\usepackage{caption}}
\AtBeginDocument{%
\ifdefined\contentsname
  \renewcommand*\contentsname{Índice}
\else
  \newcommand\contentsname{Índice}
\fi
\ifdefined\listfigurename
  \renewcommand*\listfigurename{Lista de Figuras}
\else
  \newcommand\listfigurename{Lista de Figuras}
\fi
\ifdefined\listtablename
  \renewcommand*\listtablename{Lista de Tabelas}
\else
  \newcommand\listtablename{Lista de Tabelas}
\fi
\ifdefined\figurename
  \renewcommand*\figurename{Figura}
\else
  \newcommand\figurename{Figura}
\fi
\ifdefined\tablename
  \renewcommand*\tablename{Tabela}
\else
  \newcommand\tablename{Tabela}
\fi
}
\@ifpackageloaded{float}{}{\usepackage{float}}
\floatstyle{ruled}
\@ifundefined{c@chapter}{\newfloat{codelisting}{h}{lop}}{\newfloat{codelisting}{h}{lop}[chapter]}
\floatname{codelisting}{Listagem}
\newcommand*\listoflistings{\listof{codelisting}{Lista de Listagens}}
\makeatother
\makeatletter
\makeatother
\makeatletter
\@ifpackageloaded{caption}{}{\usepackage{caption}}
\@ifpackageloaded{subcaption}{}{\usepackage{subcaption}}
\makeatother
\makeatletter
\@ifpackageloaded{sidenotes}{}{\usepackage{sidenotes}}
\@ifpackageloaded{marginnote}{}{\usepackage{marginnote}}
\makeatother

\ifLuaTeX
\usepackage[bidi=basic]{babel}
\else
\usepackage[bidi=default]{babel}
\fi
\babelprovide[main,import]{brazilian}
% get rid of language-specific shorthands (see #6817):
\let\LanguageShortHands\languageshorthands
\def\languageshorthands#1{}
\usepackage{bookmark}

\IfFileExists{xurl.sty}{\usepackage{xurl}}{} % add URL line breaks if available
\urlstyle{same} % disable monospaced font for URLs
\hypersetup{
  pdftitle={Análise de series temporais},
  pdfauthor={Rafael Sebastião Arocho e Márcio Antonio Vieira},
  pdflang={pt-br},
  colorlinks=true,
  linkcolor={blue},
  filecolor={Maroon},
  citecolor={Blue},
  urlcolor={Blue},
  pdfcreator={LaTeX via pandoc}}


\title{Análise de series temporais}
\usepackage{etoolbox}
\makeatletter
\providecommand{\subtitle}[1]{% add subtitle to \maketitle
  \apptocmd{\@title}{\par {\large #1 \par}}{}{}
}
\makeatother
\subtitle{Disciplina: Métodos Estatísticos de Previsão}
\author{Rafael Sebastião Arocho e Márcio Antonio Vieira}
\date{19/11/2025}

\begin{document}
\maketitle


\newpage{}

\subsection{Visão Geral dos Dados}\label{visuxe3o-geral-dos-dados}

Para esta análise, utilizamos o dataset \textbf{\emph{House Property
Sales Time Series}}.

\begin{itemize}
\item
  \textbf{Fonte dos dados:}
  \href{https://www.kaggle.com/datasets/htagholdings/property-sales?select=ma_lga_12345.csv}{Kaggle.}
\item
  \textbf{Período:} 2007 a 2019
\item
  \textbf{Frequência:} Trimestral (\(n = 347\) observações)
\item
  \textbf{Variável Resposta (}\(Y_t\)):

  \begin{itemize}
  \tightlist
  \item
    \emph{MA}: Preço (\$) mediano de casas e unidades habitacionais
  \end{itemize}
\end{itemize}

\begin{table}

\caption{\label{tbl-resumo-serie}{[}``resumo dos dados tbl1'',``resumo
dos dados tbl2''{]}}

\begin{minipage}{\linewidth}

\fontsize{12.0pt}{14.0pt}\selectfont
\begin{tabular*}{\linewidth}{@{\extracolsep{\fill}}lccccc}
\toprule
\textbf{Characteristic} & \textbf{N} & \textbf{Overall}  N = 247\textsuperscript{\textit{1}} & \textbf{1}  N = 48\textsuperscript{\textit{1}} & \textbf{2}  N = 99\textsuperscript{\textit{1}} & \textbf{3}  N = 100\textsuperscript{\textit{1}} \\ 
\midrule\addlinespace[2.5pt]
MA & 247 & 456,822 (412,509, 518,911) & 332,876 (326,076, 338,709) & 432,801 (425,751, 469,920) & 540,086 (507,052, 601,104) \\ 
type & 247 &  &  &  &  \\ 
    house &  & 100 (40\%) & 0 (0\%) & 49 (49\%) & 51 (51\%) \\ 
    unit &  & 147 (60\%) & 48 (100\%) & 50 (51\%) & 49 (49\%) \\ 
\bottomrule
\end{tabular*}
\begin{minipage}{\linewidth}
\textsuperscript{\textit{1}}Median (Q1, Q3); n (\%)\\
\end{minipage}

\end{minipage}%
\newline
\begin{minipage}{\linewidth}

\fontsize{12.0pt}{14.0pt}\selectfont
\begin{tabular*}{\linewidth}{@{\extracolsep{\fill}}lcccc}
\toprule
\textbf{Characteristic} & \textbf{N} & \textbf{Overall}  N = 100\textsuperscript{\textit{1}} & \textbf{4}  N = 51\textsuperscript{\textit{1}} & \textbf{5}  N = 49\textsuperscript{\textit{1}} \\ 
\midrule\addlinespace[2.5pt]
MA & 100 & 771,248 (636,635, 824,491) & 636,687 (599,614, 745,430) & 807,826 (775,199, 952,327) \\ 
type & 100 &  &  &  \\ 
    house &  & 100 (100\%) & 51 (100\%) & 49 (100\%) \\ 
\bottomrule
\end{tabular*}
\begin{minipage}{\linewidth}
\textsuperscript{\textit{1}}Median (Q1, Q3); n (\%)\\
\end{minipage}

\end{minipage}%

\end{table}%

\newpage{}

\subsection{Visualização da Série}\label{visualizauxe7uxe3o-da-suxe9rie}

\begin{figure}

\begin{minipage}{0.50\linewidth}

\begin{figure}[H]

{\centering \pandocbounded{\includegraphics[keepaspectratio]{trabalho_mep_files/figure-pdf/gráfico da série-1.pdf}}

}

\subcaption{serie completa}

\end{figure}%

\end{minipage}%
%
\begin{minipage}{0.50\linewidth}

\begin{figure}[H]

{\centering \pandocbounded{\includegraphics[keepaspectratio]{trabalho_mep_files/figure-pdf/gráfico da série-2.pdf}}

}

\subcaption{serie decomposta}

\end{figure}%

\end{minipage}%
\newline
\begin{minipage}{0.50\linewidth}
distribuições das series\end{minipage}%

\end{figure}%

Observamos que será necessário separar as series em tipo de moradia e
número de quartos para simplificar o ajuste.\\
Parece haver tendência de crescimento em praticamente todas as séries
decompostas.\\
Vamos utilizar diferenciação de nível um e dois para ter outro olhar
sobre as séries.

\begin{figure*}[H]

\begin{figure}[H]

{\centering \pandocbounded{\includegraphics[keepaspectratio]{trabalho_mep_files/figure-pdf/unnamed-chunk-1-1.pdf}}

}

\caption{diferenciação de primeiro nível}

\end{figure}%

\end{figure*}

\begin{figure*}[H]

\begin{figure}[H]

{\centering \pandocbounded{\includegraphics[keepaspectratio]{trabalho_mep_files/figure-pdf/unnamed-chunk-1-2.pdf}}

}

\caption{diferenciação de segundo nível}

\end{figure}%

\end{figure*}

comportamentos diferenciados

Observando as séries com as diferenciações aplicadas vamos seguir a
análise com os preços de diferentes combinações de tipo de domicilio e
número de quartos.

\newpage{} \#\# Ajuste de modelos

\subsubsection{type=house x bedrooms=5}\label{typehouse-x-bedrooms5}

Para a série de type: \emph{house} e 5 quartos testamos diferenciação
para tornar a serie estacionária.

\begin{Shaded}
\begin{Highlighting}[]
\NormalTok{an\_1 }\OtherTok{\textless{}{-}} \FunctionTok{plot\_acf\_pacf}\NormalTok{(}\AttributeTok{data=}\NormalTok{treino, }\AttributeTok{type=}\StringTok{"house"}\NormalTok{, }\AttributeTok{bedrooms=}\DecValTok{5}\NormalTok{, }\AttributeTok{difference=}\DecValTok{1}\NormalTok{)}
\NormalTok{an\_1}
\end{Highlighting}
\end{Shaded}

\begin{verbatim}
[[1]]
\end{verbatim}

\begin{figure}[H]

{\centering \pandocbounded{\includegraphics[keepaspectratio]{trabalho_mep_files/figure-pdf/acf_pacf primeiro modelo-1.pdf}}

}

\caption{ACF}

\end{figure}%

\begin{verbatim}

[[2]]
\end{verbatim}

\begin{figure}[H]

{\centering \pandocbounded{\includegraphics[keepaspectratio]{trabalho_mep_files/figure-pdf/acf_pacf primeiro modelo-2.pdf}}

}

\caption{PACF}

\end{figure}%

\begin{Shaded}
\begin{Highlighting}[]
\NormalTok{an\_2 }\OtherTok{\textless{}{-}} \FunctionTok{plot\_acf\_pacf}\NormalTok{(}\AttributeTok{data=}\NormalTok{treino, }\AttributeTok{type=}\StringTok{"house"}\NormalTok{, }\AttributeTok{bedrooms=}\DecValTok{5}\NormalTok{, }\AttributeTok{difference=}\DecValTok{2}\NormalTok{)}
\NormalTok{an\_2}
\end{Highlighting}
\end{Shaded}

\begin{verbatim}
[[1]]
\end{verbatim}

\begin{figure}[H]

{\centering \pandocbounded{\includegraphics[keepaspectratio]{trabalho_mep_files/figure-pdf/acf_pacf primeiro modelo-3.pdf}}

}

\caption{ACF}

\end{figure}%

\begin{verbatim}

[[2]]
\end{verbatim}

\begin{figure}[H]

{\centering \pandocbounded{\includegraphics[keepaspectratio]{trabalho_mep_files/figure-pdf/acf_pacf primeiro modelo-4.pdf}}

}

\caption{PACF}

\end{figure}%

Parecemos ter um comportamento de ondas senóides no gráfico de ACF e um
comportamento irregular no PACF que pode indicar sazonalidade. Vamos
ajustar de ínicio um modelo ARIMA(2, 1, 0).\\
Após o ajuste inicial faremos a sobrefixação dos modelos para
identificar o melhor modelo através da avaliação das métricas de AIC

\begin{verbatim}
# A tibble: 3 x 8
  type  bedrooms .model               term  estimate std.error statistic p.value
  <chr>    <int> <chr>                <chr>    <dbl>     <dbl>     <dbl>   <dbl>
1 house        5 ARIMA(MA ~ pdq(2, 1~ ar1      0.356     0.148      2.40 0.0208 
2 house        5 ARIMA(MA ~ pdq(2, 1~ ar2      0.216     0.149      1.45 0.154  
3 house        5 ARIMA(MA ~ pdq(2, 1~ cons~ 2707.      907.         2.98 0.00472
\end{verbatim}

\begin{verbatim}
# A tibble: 2 x 8
  type  bedrooms .model               term  estimate std.error statistic p.value
  <chr>    <int> <chr>                <chr>    <dbl>     <dbl>     <dbl>   <dbl>
1 house        5 ARIMA(MA ~ pdq(1, 1~ ar1      0.454     0.136      3.34 1.79e-3
2 house        5 ARIMA(MA ~ pdq(1, 1~ cons~ 3544.      949.         3.74 5.59e-4
\end{verbatim}

\begin{verbatim}
# A tibble: 3 x 8
  type  bedrooms .model               term  estimate std.error statistic p.value
  <chr>    <int> <chr>                <chr>    <dbl>     <dbl>     <dbl>   <dbl>
1 house        5 ARIMA(MA ~ pdq(1, 1~ ar1      0.811     0.189      4.30 1.00e-4
2 house        5 ARIMA(MA ~ pdq(1, 1~ ma1     -0.485     0.311     -1.56 1.26e-1
3 house        5 ARIMA(MA ~ pdq(1, 1~ cons~ 1192.      452.         2.64 1.17e-2
\end{verbatim}

\begin{verbatim}
# A tibble: 1 x 8
  type  bedrooms .model               term  estimate std.error statistic p.value
  <chr>    <int> <chr>                <chr>    <dbl>     <dbl>     <dbl>   <dbl>
1 house        5 ARIMA(MA ~ pdq(1, 2~ ar1     -0.425     0.138     -3.07 0.00383
\end{verbatim}

\begin{verbatim}
# A tibble: 2 x 8
  type  bedrooms .model               term  estimate std.error statistic p.value
  <chr>    <int> <chr>                <chr>    <dbl>     <dbl>     <dbl>   <dbl>
1 house        5 ARIMA(MA ~ pdq(2, 2~ ar1     -0.485     0.153    -3.17  0.00289
2 house        5 ARIMA(MA ~ pdq(2, 2~ ar2     -0.136     0.151    -0.898 0.374  
\end{verbatim}

\begin{verbatim}
# A tibble: 2 x 8
  type  bedrooms .model               term  estimate std.error statistic p.value
  <chr>    <int> <chr>                <chr>    <dbl>     <dbl>     <dbl>   <dbl>
1 house        5 ARIMA(MA ~ pdq(0, 2~ ma1     -0.450     0.155     -2.91 0.00584
2 house        5 ARIMA(MA ~ pdq(0, 2~ sma1    -0.756     0.217     -3.49 0.00118
\end{verbatim}

\begin{verbatim}
# A tibble: 3 x 8
  type  bedrooms .model               term  estimate std.error statistic p.value
  <chr>    <int> <chr>                <chr>    <dbl>     <dbl>     <dbl>   <dbl>
1 house        5 ARIMA(MA ~ pdq(1, 2~ ar1    1.36e-4     0.295  0.000462 1.000  
2 house        5 ARIMA(MA ~ pdq(1, 2~ ma1   -4.50e-1     0.278 -1.62     0.113  
3 house        5 ARIMA(MA ~ pdq(1, 2~ sma1  -7.57e-1     0.222 -3.41     0.00148
\end{verbatim}

Ajustados os modelos daremos sequência na análise com o modelo
ARIMA(0,2,1)(0,0,1){[}4{]} por ter apresentado significância nos
parâmetros e o menor dos valores de AIC

\begin{Shaded}
\begin{Highlighting}[]
\NormalTok{ggtime}\SpecialCharTok{::}\FunctionTok{gg\_tsresiduals}\NormalTok{(modelo\_arima\_6)}
\end{Highlighting}
\end{Shaded}

\pandocbounded{\includegraphics[keepaspectratio]{trabalho_mep_files/figure-pdf/unnamed-chunk-4-1.pdf}}

\begin{Shaded}
\begin{Highlighting}[]
\FunctionTok{residuals}\NormalTok{(modelo\_arima\_6) }\SpecialCharTok{|\textgreater{}} 
  \FunctionTok{ggplot}\NormalTok{(}\AttributeTok{mapping=}\FunctionTok{aes}\NormalTok{(}\AttributeTok{x=}\NormalTok{trimestre, }\AttributeTok{y=}\NormalTok{.resid)) }\SpecialCharTok{+} 
  \FunctionTok{geom\_jitter}\NormalTok{() }\SpecialCharTok{+}
  \FunctionTok{theme\_minimal}\NormalTok{()}
\end{Highlighting}
\end{Shaded}

\pandocbounded{\includegraphics[keepaspectratio]{trabalho_mep_files/figure-pdf/unnamed-chunk-4-2.pdf}}

\begin{Shaded}
\begin{Highlighting}[]
\FunctionTok{shapiro.test}\NormalTok{(}\FunctionTok{residuals}\NormalTok{(modelo\_arima\_6)}\SpecialCharTok{$}\NormalTok{.resid)}
\end{Highlighting}
\end{Shaded}

\begin{verbatim}

    Shapiro-Wilk normality test

data:  residuals(modelo_arima_6)$.resid
W = 0.96361, p-value = 0.1877
\end{verbatim}

\begin{Shaded}
\begin{Highlighting}[]
\FunctionTok{Box.test}\NormalTok{(}\FunctionTok{residuals}\NormalTok{(modelo\_arima\_6)}\SpecialCharTok{$}\NormalTok{.resid, }\AttributeTok{lag=}\DecValTok{12}\NormalTok{)}
\end{Highlighting}
\end{Shaded}

\begin{verbatim}

    Box-Pierce test

data:  residuals(modelo_arima_6)$.resid
X-squared = 9.3267, df = 12, p-value = 0.6748
\end{verbatim}

Os testes nos mostram que os residuos apresentam normalidade e o teste
de box-pierce não rejeita a hipótese de independencia, o que é um
indicador que o ajuste está adequado.

Vamos fazer uma previsão alguns passos a frente

\begin{Shaded}
\begin{Highlighting}[]
\FunctionTok{forecast}\NormalTok{(modelo\_arima\_6, }\AttributeTok{h =} \DecValTok{6}\NormalTok{, }\AttributeTok{level =} \FunctionTok{c}\NormalTok{(}\DecValTok{80}\NormalTok{, }\DecValTok{95}\NormalTok{)) }\SpecialCharTok{\%\textgreater{}\%}
  \FunctionTok{autoplot}\NormalTok{(}\FunctionTok{bind\_rows}\NormalTok{(treino\_casa\_5, teste\_casa\_5)) }\SpecialCharTok{+}
  \FunctionTok{labs}\NormalTok{(}
    \AttributeTok{title =} \StringTok{"Previsão ARIMA(0,2,1)(0,0,1)[4]"}\NormalTok{,}
    \AttributeTok{subtitle =} \StringTok{"Horizonte: 6 trimestres | Intervalos: 80\% e 95\%"}\NormalTok{,}
    \AttributeTok{y =} \StringTok{"Preço Mediano"}\NormalTok{,}
    \AttributeTok{x =} \StringTok{"Trimestre"}
\NormalTok{  ) }\SpecialCharTok{+} \FunctionTok{theme\_classic}\NormalTok{()}
\end{Highlighting}
\end{Shaded}

\pandocbounded{\includegraphics[keepaspectratio]{trabalho_mep_files/figure-pdf/unnamed-chunk-5-1.pdf}}

\paragraph{Modelo de alisamento
exponencial}\label{modelo-de-alisamento-exponencial}

Seguindo a mesma lógica dos ajustes anteriores faremos o ajuste de
modelo de alisamento exponencial. Por não termos identificado
sazonalidade utilizaremos alisamento exponencial de Holt-Winters aditivo
por haver tendência e sazonalidade na série

\begin{Shaded}
\begin{Highlighting}[]
\NormalTok{ma }\OtherTok{\textless{}{-}}  \FunctionTok{ts}\NormalTok{(treino\_casa\_5}\SpecialCharTok{$}\NormalTok{MA, }\AttributeTok{start =} \FunctionTok{c}\NormalTok{(}\DecValTok{2007}\NormalTok{, }\DecValTok{3}\NormalTok{), }\AttributeTok{frequency =} \DecValTok{4}\NormalTok{)}

\NormalTok{AEH }\OtherTok{\textless{}{-}} \FunctionTok{HoltWinters}\NormalTok{(ma, }\AttributeTok{alpha =} \ConstantTok{NULL}\NormalTok{, }\AttributeTok{beta =} \ConstantTok{NULL}\NormalTok{, }\AttributeTok{gamma =} \ConstantTok{TRUE}\NormalTok{, }\AttributeTok{seasonal =} \StringTok{"additive"}\NormalTok{)}
\CommentTok{\# Calculo das previsoes 6 passos a frente e os intervalos de previsao}
\NormalTok{previsao }\OtherTok{=} \FunctionTok{predict}\NormalTok{(AEH, }
                   \AttributeTok{n.ahead=}\DecValTok{6}\NormalTok{, }
                   \AttributeTok{prediction.interval =} \ConstantTok{TRUE}\NormalTok{, }
                   \AttributeTok{level =} \FloatTok{0.95}\NormalTok{, }
                   \AttributeTok{interval=}\StringTok{"prediction"}\NormalTok{) }
\NormalTok{previsao}
\end{Highlighting}
\end{Shaded}

\begin{verbatim}
            fit     upr     lwr
2018 Q2 1025381 1044152 1006609
2018 Q3 1035293 1061462 1009125
2018 Q4 1046315 1079847 1012783
2019 Q1 1056702 1097765 1015638
2019 Q2 1066553 1117241 1015866
2019 Q3 1076466 1134950 1017982
\end{verbatim}

\begin{Shaded}
\begin{Highlighting}[]
\CommentTok{\# Constroi o grafico com ajuste, previsoes e intervalos de previsao}
\FunctionTok{plot}\NormalTok{(AEH, previsao, }\AttributeTok{lwd=}\DecValTok{2}\NormalTok{, }\AttributeTok{col=}\StringTok{"black"}\NormalTok{, }\AttributeTok{xlab=}\StringTok{"Ano"}\NormalTok{, }\AttributeTok{ylab=}\ConstantTok{NA}\NormalTok{)}
\end{Highlighting}
\end{Shaded}

\pandocbounded{\includegraphics[keepaspectratio]{trabalho_mep_files/figure-pdf/unnamed-chunk-6-1.pdf}}

Faremos agora a comparação dos dois modelos utilizando erro quadrático
médio

\begin{Shaded}
\begin{Highlighting}[]
\CommentTok{\# As previsões pontuais são a primeira coluna da matriz de previsão:}
\NormalTok{previsoes\_pontuais\_hw }\OtherTok{\textless{}{-}}\NormalTok{ previsao[,}\StringTok{"fit"}\NormalTok{]}
\NormalTok{previsoes\_pontuais\_arima }\OtherTok{\textless{}{-}} \FunctionTok{forecast}\NormalTok{(modelo\_arima\_6, }\AttributeTok{h=}\DecValTok{6}\NormalTok{)}\SpecialCharTok{$}\NormalTok{.mean}
\NormalTok{valores\_reais }\OtherTok{\textless{}{-}}\NormalTok{ teste\_casa\_5}\SpecialCharTok{$}\NormalTok{MA}

\CommentTok{\# Cálculo do Erro Quadrático Médio (MSE)}
\NormalTok{erros\_hw }\OtherTok{\textless{}{-}}\NormalTok{ previsoes\_pontuais\_hw }\SpecialCharTok{{-}}\NormalTok{ valores\_reais}
\NormalTok{erros\_arima }\OtherTok{\textless{}{-}}\NormalTok{ previsoes\_pontuais\_arima }\SpecialCharTok{{-}}\NormalTok{ valores\_reais}
\NormalTok{MSE\_hw }\OtherTok{\textless{}{-}} \FunctionTok{mean}\NormalTok{(erros\_hw}\SpecialCharTok{\^{}}\DecValTok{2}\NormalTok{)}
\NormalTok{MSE\_arima }\OtherTok{\textless{}{-}} \FunctionTok{mean}\NormalTok{(erros\_arima}\SpecialCharTok{\^{}}\DecValTok{2}\NormalTok{)}
\FunctionTok{cat}\NormalTok{(}\StringTok{"}\SpecialCharTok{\textbackslash{}n}\StringTok{Previsões Pontuais HW:}\SpecialCharTok{\textbackslash{}n}\StringTok{"}\NormalTok{)}
\end{Highlighting}
\end{Shaded}

\begin{verbatim}

Previsões Pontuais HW:
\end{verbatim}

\begin{Shaded}
\begin{Highlighting}[]
\FunctionTok{print}\NormalTok{(previsoes\_pontuais\_hw)}
\end{Highlighting}
\end{Shaded}

\begin{verbatim}
        Qtr1    Qtr2    Qtr3    Qtr4
2018         1025381 1035293 1046315
2019 1056702 1066553 1076466        
\end{verbatim}

\begin{Shaded}
\begin{Highlighting}[]
\FunctionTok{cat}\NormalTok{(}\StringTok{"}\SpecialCharTok{\textbackslash{}n}\StringTok{Previsões Pontuais ARIMA:}\SpecialCharTok{\textbackslash{}n}\StringTok{"}\NormalTok{)}
\end{Highlighting}
\end{Shaded}

\begin{verbatim}

Previsões Pontuais ARIMA:
\end{verbatim}

\begin{Shaded}
\begin{Highlighting}[]
\FunctionTok{print}\NormalTok{(previsoes\_pontuais\_arima)}
\end{Highlighting}
\end{Shaded}

\begin{verbatim}
[1] 1022220 1026582 1033842 1044259 1052874 1061490
\end{verbatim}

\begin{Shaded}
\begin{Highlighting}[]
\FunctionTok{cat}\NormalTok{(}\StringTok{"}\SpecialCharTok{\textbackslash{}n}\StringTok{Valores Reais (Base de Teste):}\SpecialCharTok{\textbackslash{}n}\StringTok{"}\NormalTok{)}
\end{Highlighting}
\end{Shaded}

\begin{verbatim}

Valores Reais (Base de Teste):
\end{verbatim}

\begin{Shaded}
\begin{Highlighting}[]
\FunctionTok{print}\NormalTok{(valores\_reais)}
\end{Highlighting}
\end{Shaded}

\begin{verbatim}
[1] 1017752 1007114 1002323  998136  995363  970268
\end{verbatim}

\begin{Shaded}
\begin{Highlighting}[]
\FunctionTok{cat}\NormalTok{(}\StringTok{"}\SpecialCharTok{\textbackslash{}n}\StringTok{Erro Quadrático Médio (MSE) para os 6 passos HW:}\SpecialCharTok{\textbackslash{}n}\StringTok{"}\NormalTok{)}
\end{Highlighting}
\end{Shaded}

\begin{verbatim}

Erro Quadrático Médio (MSE) para os 6 passos HW:
\end{verbatim}

\begin{Shaded}
\begin{Highlighting}[]
\FunctionTok{print}\NormalTok{(MSE\_hw)}
\end{Highlighting}
\end{Shaded}

\begin{verbatim}
[1] 3760607450
\end{verbatim}

\begin{Shaded}
\begin{Highlighting}[]
\FunctionTok{cat}\NormalTok{(}\StringTok{"}\SpecialCharTok{\textbackslash{}n}\StringTok{Erro Quadrático Médio (MSE) para os 6 passos ARIMA:}\SpecialCharTok{\textbackslash{}n}\StringTok{"}\NormalTok{)}
\end{Highlighting}
\end{Shaded}

\begin{verbatim}

Erro Quadrático Médio (MSE) para os 6 passos ARIMA:
\end{verbatim}

\begin{Shaded}
\begin{Highlighting}[]
\FunctionTok{print}\NormalTok{(MSE\_arima)}
\end{Highlighting}
\end{Shaded}

\begin{verbatim}
[1] 2524795264
\end{verbatim}

Observamos um EQM menor no ajuste com o modelo ARIMA, então podemos
concluir que se trata do melhor modelo para ajustar a série de preços
medianos de casas com 5 quartos.




\end{document}
